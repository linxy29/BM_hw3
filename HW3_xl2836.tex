\documentclass[]{article}
\usepackage{lmodern}
\usepackage{amssymb,amsmath}
\usepackage{ifxetex,ifluatex}
\usepackage{fixltx2e} % provides \textsubscript
\ifnum 0\ifxetex 1\fi\ifluatex 1\fi=0 % if pdftex
  \usepackage[T1]{fontenc}
  \usepackage[utf8]{inputenc}
\else % if luatex or xelatex
  \ifxetex
    \usepackage{mathspec}
  \else
    \usepackage{fontspec}
  \fi
  \defaultfontfeatures{Ligatures=TeX,Scale=MatchLowercase}
\fi
% use upquote if available, for straight quotes in verbatim environments
\IfFileExists{upquote.sty}{\usepackage{upquote}}{}
% use microtype if available
\IfFileExists{microtype.sty}{%
\usepackage{microtype}
\UseMicrotypeSet[protrusion]{basicmath} % disable protrusion for tt fonts
}{}
\usepackage[margin=1in]{geometry}
\usepackage{hyperref}
\hypersetup{unicode=true,
            pdftitle={HW3\_xl2836},
            pdfauthor={Xinyi Lin},
            pdfborder={0 0 0},
            breaklinks=true}
\urlstyle{same}  % don't use monospace font for urls
\usepackage{color}
\usepackage{fancyvrb}
\newcommand{\VerbBar}{|}
\newcommand{\VERB}{\Verb[commandchars=\\\{\}]}
\DefineVerbatimEnvironment{Highlighting}{Verbatim}{commandchars=\\\{\}}
% Add ',fontsize=\small' for more characters per line
\usepackage{framed}
\definecolor{shadecolor}{RGB}{248,248,248}
\newenvironment{Shaded}{\begin{snugshade}}{\end{snugshade}}
\newcommand{\KeywordTok}[1]{\textcolor[rgb]{0.13,0.29,0.53}{\textbf{#1}}}
\newcommand{\DataTypeTok}[1]{\textcolor[rgb]{0.13,0.29,0.53}{#1}}
\newcommand{\DecValTok}[1]{\textcolor[rgb]{0.00,0.00,0.81}{#1}}
\newcommand{\BaseNTok}[1]{\textcolor[rgb]{0.00,0.00,0.81}{#1}}
\newcommand{\FloatTok}[1]{\textcolor[rgb]{0.00,0.00,0.81}{#1}}
\newcommand{\ConstantTok}[1]{\textcolor[rgb]{0.00,0.00,0.00}{#1}}
\newcommand{\CharTok}[1]{\textcolor[rgb]{0.31,0.60,0.02}{#1}}
\newcommand{\SpecialCharTok}[1]{\textcolor[rgb]{0.00,0.00,0.00}{#1}}
\newcommand{\StringTok}[1]{\textcolor[rgb]{0.31,0.60,0.02}{#1}}
\newcommand{\VerbatimStringTok}[1]{\textcolor[rgb]{0.31,0.60,0.02}{#1}}
\newcommand{\SpecialStringTok}[1]{\textcolor[rgb]{0.31,0.60,0.02}{#1}}
\newcommand{\ImportTok}[1]{#1}
\newcommand{\CommentTok}[1]{\textcolor[rgb]{0.56,0.35,0.01}{\textit{#1}}}
\newcommand{\DocumentationTok}[1]{\textcolor[rgb]{0.56,0.35,0.01}{\textbf{\textit{#1}}}}
\newcommand{\AnnotationTok}[1]{\textcolor[rgb]{0.56,0.35,0.01}{\textbf{\textit{#1}}}}
\newcommand{\CommentVarTok}[1]{\textcolor[rgb]{0.56,0.35,0.01}{\textbf{\textit{#1}}}}
\newcommand{\OtherTok}[1]{\textcolor[rgb]{0.56,0.35,0.01}{#1}}
\newcommand{\FunctionTok}[1]{\textcolor[rgb]{0.00,0.00,0.00}{#1}}
\newcommand{\VariableTok}[1]{\textcolor[rgb]{0.00,0.00,0.00}{#1}}
\newcommand{\ControlFlowTok}[1]{\textcolor[rgb]{0.13,0.29,0.53}{\textbf{#1}}}
\newcommand{\OperatorTok}[1]{\textcolor[rgb]{0.81,0.36,0.00}{\textbf{#1}}}
\newcommand{\BuiltInTok}[1]{#1}
\newcommand{\ExtensionTok}[1]{#1}
\newcommand{\PreprocessorTok}[1]{\textcolor[rgb]{0.56,0.35,0.01}{\textit{#1}}}
\newcommand{\AttributeTok}[1]{\textcolor[rgb]{0.77,0.63,0.00}{#1}}
\newcommand{\RegionMarkerTok}[1]{#1}
\newcommand{\InformationTok}[1]{\textcolor[rgb]{0.56,0.35,0.01}{\textbf{\textit{#1}}}}
\newcommand{\WarningTok}[1]{\textcolor[rgb]{0.56,0.35,0.01}{\textbf{\textit{#1}}}}
\newcommand{\AlertTok}[1]{\textcolor[rgb]{0.94,0.16,0.16}{#1}}
\newcommand{\ErrorTok}[1]{\textcolor[rgb]{0.64,0.00,0.00}{\textbf{#1}}}
\newcommand{\NormalTok}[1]{#1}
\usepackage{longtable,booktabs}
\usepackage{graphicx,grffile}
\makeatletter
\def\maxwidth{\ifdim\Gin@nat@width>\linewidth\linewidth\else\Gin@nat@width\fi}
\def\maxheight{\ifdim\Gin@nat@height>\textheight\textheight\else\Gin@nat@height\fi}
\makeatother
% Scale images if necessary, so that they will not overflow the page
% margins by default, and it is still possible to overwrite the defaults
% using explicit options in \includegraphics[width, height, ...]{}
\setkeys{Gin}{width=\maxwidth,height=\maxheight,keepaspectratio}
\IfFileExists{parskip.sty}{%
\usepackage{parskip}
}{% else
\setlength{\parindent}{0pt}
\setlength{\parskip}{6pt plus 2pt minus 1pt}
}
\setlength{\emergencystretch}{3em}  % prevent overfull lines
\providecommand{\tightlist}{%
  \setlength{\itemsep}{0pt}\setlength{\parskip}{0pt}}
\setcounter{secnumdepth}{0}
% Redefines (sub)paragraphs to behave more like sections
\ifx\paragraph\undefined\else
\let\oldparagraph\paragraph
\renewcommand{\paragraph}[1]{\oldparagraph{#1}\mbox{}}
\fi
\ifx\subparagraph\undefined\else
\let\oldsubparagraph\subparagraph
\renewcommand{\subparagraph}[1]{\oldsubparagraph{#1}\mbox{}}
\fi

%%% Use protect on footnotes to avoid problems with footnotes in titles
\let\rmarkdownfootnote\footnote%
\def\footnote{\protect\rmarkdownfootnote}

%%% Change title format to be more compact
\usepackage{titling}

% Create subtitle command for use in maketitle
\newcommand{\subtitle}[1]{
  \posttitle{
    \begin{center}\large#1\end{center}
    }
}

\setlength{\droptitle}{-2em}

  \title{HW3\_xl2836}
    \pretitle{\vspace{\droptitle}\centering\huge}
  \posttitle{\par}
    \author{Xinyi Lin}
    \preauthor{\centering\large\emph}
  \postauthor{\par}
      \predate{\centering\large\emph}
  \postdate{\par}
    \date{10/4/2018}


\begin{document}
\maketitle

\begin{Shaded}
\begin{Highlighting}[]
\KeywordTok{library}\NormalTok{(tidyverse)}
\end{Highlighting}
\end{Shaded}

\begin{verbatim}
## -- Attaching packages -------------------------------------------------------------- tidyverse 1.2.1 --
\end{verbatim}

\begin{verbatim}
## √ ggplot2 3.0.0     √ purrr   0.2.5
## √ tibble  1.4.2     √ dplyr   0.7.6
## √ tidyr   0.8.1     √ stringr 1.3.1
## √ readr   1.1.1     √ forcats 0.3.0
\end{verbatim}

\begin{verbatim}
## -- Conflicts ----------------------------------------------------------------- tidyverse_conflicts() --
## x dplyr::filter() masks stats::filter()
## x dplyr::lag()    masks stats::lag()
\end{verbatim}

\begin{Shaded}
\begin{Highlighting}[]
\KeywordTok{library}\NormalTok{(readxl)}
\end{Highlighting}
\end{Shaded}

\section{Problem 1}\label{problem-1}

\subsection{Question 1}\label{question-1}

Assuming the mean of \(X_i\) is \(\mu\).

\begin{split}
E(s^2) & =E[\frac{1}{n-1}\sum_{i=1}^n(x_i-x)^2]\\
& =E[\frac{1}{n-1}\sum_{i=1}^n(x_i^2-2\bar{x}x_i+\bar{x}^2)] \\
& =\frac{n}{n-1}E[\frac{1}{n}\sum_{i=1}^n(x_i^2-2\bar{x}x_i+\bar{x}^2)] \\
& =\frac{n}{n-1}\{\frac{1}{n}E[\sum_{i=1}^n(x_i^2)]-2E(\bar{x}\frac{1}{n}\sum_{i=1}^nx_i)+E(\bar{x}^2)\} \\
& =\frac{n}{n-1}\{\frac{1}{n}\sum_{i=1}^nE(x_i^2)-2E(\bar{x}^2)+E(\bar{x}^2)\} \\
& =\frac{n}{n-1}\{\frac{1}{n}\sum_{i=1}^nE(x_i^2)-E(\bar{x}^2)\}
\end{split}

As \(Var(x_i) = E(x_i^2)-(Ex_i)^2\), we can get
\(E(x_i^2)=Var(x_i)+(Ex_i)^2=\sigma^2+\mu^2\). As
\(Var(\bar{x}) = E(\bar{x}^2)-(E\bar{x})^2\), we can get
\(E(\bar{x}^2)=Var(\bar{x})+(E\bar{x})^2=\frac{\sigma^2}{n}+\mu^2\).
Then,

\begin{split}
E(s^2) & =\frac{n}{n-1}[\frac{1}{n}\times n(\sigma^2+\mu^2)-\frac{\sigma^2}{n}-\mu^2] \\
& =\frac{n}{n-1}\frac{1}{n}\sigma^2 \\
& =\sigma^2
\end{split}

\subsection{Question 2}\label{question-2}

\begin{split}
\sum_{i=1}^k\sum_{j=1}^{n_i}(y_{ij}-\bar{\bar{y}})^2 & =\sum_{i=1}^k\sum_{j=1}^{n_i}(y_{ij}^2-2y_{ij}\bar{\bar{y}}+\bar{\bar{y}}^2) \\
& =\sum_{i=1}^k\sum_{j=1}^{n_i}(y_{ij}^2+\bar{\bar{y}}^2)-2\bar{\bar{y}}\sum_{i=1}^k\sum_{j=1}^{n_i}y_{ij} \\
& =\sum_{i=1}^k\sum_{j=1}^{n_i}(y_{ij}^2+\bar{\bar{y}}^2)-2\sum_{i=1}^k\sum_{j=1}^{n_i}\bar{\bar{y}}^2 \\
& =\sum_{i=1}^k\sum_{j=1}^{n_i}(y_{ij}^2-\bar{\bar{y}}^2)
\end{split}

\begin{split}
\sum_{i=1}^k\sum_{j=1}^{n_i}(y_{ij}-\bar{y_l})^2+\sum_{i=1}^k\sum_{j=1}^{n_i}(\bar{y_l}-\bar{\bar{y}})^2 & =\sum_{i=1}^k\sum_{j=1}^{n_i}[y_{ij}^2-2\bar{y_l}y_{ij}+(\bar{y_l})^2+(\bar{y_l})^2-2\bar{y_l}\bar{\bar{y}}+\bar{\bar{y}}^2] \\
& =\sum_{i=1}^k\sum_{j=1}^{n_i}(y_{ij}^2+\bar{\bar{y}}^2)+\sum_{i=1}^k\sum_{j=1}^{n_i}[2(\bar{y_l})^2-2\bar{y_l}\bar{\bar{y}}+\bar{\bar{y}}^2] \\
& =\sum_{i=1}^k\sum_{j=1}^{n_i}(y_{ij}^2+\bar{\bar{y}}^2)+2\sum_{i=1}^k\sum_{j=1}^{n_i}\bar{y_l}^2-2\sum_{i=1}^k\sum_{j=1}^{n_i}\bar{y_l}y_{ij}-2\sum_{i=1}^k\sum_{j=1}^{n_i}\bar{y_l}\bar{\bar{y}} \\
& =\sum_{i=1}^k\sum_{j=1}^{n_i}(y_{ij}^2+\bar{\bar{y}}^2)+2\sum_{i=1}^k\sum_{j=1}^{n_i}\bar{y_l}^2-2\sum_{i=1}^k(\bar{y_l}\sum_{j=1}^{n_i}y_{ij})-2\bar{\bar{y}}\sum_{i=1}^k(\bar{y_l}\sum_{j=1}^{n_i}1) \\
& =\sum_{i=1}^k\sum_{j=1}^{n_i}(y_{ij}^2+\bar{\bar{y}}^2)+2\sum_{i=1}^k\sum_{j=1}^{n_i}\bar{y_l}^2-2\sum_{i=1}^k\sum_{j=1}^{n_i}\bar{y_l}^2-2\sum_{i=1}^k\sum_{j=1}^{n_i}\bar{\bar{y}}^2 \\
& =\sum_{i=1}^k\sum_{j=1}^{n_i}(y_{ij}^2-\bar{\bar{y}}^2) \\
& =\sum_{i=1}^k\sum_{j=1}^{n_i}(y_{ij}-\bar{\bar{y}})^2

\end{split}

\section{Problem 2}\label{problem-2}

First, we need to import data ``HeavySmoke.csv'' and ``NeverSmoke.csv''.

\begin{Shaded}
\begin{Highlighting}[]
\NormalTok{heavysmoke_df =}\StringTok{ }\KeywordTok{read_csv}\NormalTok{(}\DataTypeTok{file =} \StringTok{"./data/HeavySmoke.csv"}\NormalTok{)}
\end{Highlighting}
\end{Shaded}

\begin{verbatim}
## Parsed with column specification:
## cols(
##   ID = col_integer(),
##   BMI_base = col_double(),
##   BMI_6yrs = col_double()
## )
\end{verbatim}

\begin{Shaded}
\begin{Highlighting}[]
\NormalTok{neversmoke_df =}\StringTok{ }\KeywordTok{read_csv}\NormalTok{(}\DataTypeTok{file =} \StringTok{"./data/NeverSmoke.csv"}\NormalTok{)}
\end{Highlighting}
\end{Shaded}

\begin{verbatim}
## Parsed with column specification:
## cols(
##   ID = col_integer(),
##   BMI_base = col_double(),
##   BMI_6yrs = col_double()
## )
\end{verbatim}

\subsection{Question 1}\label{question-1-1}

In order to test wether BMI has changed 6 years after quitting smoking,
we need to test the means of \texttt{BMI\_base} and \texttt{BMI\_6yrs}
are different or not. As we don't know the variance of two samples, we
use t test.

Assuming the mean of \texttt{BMI\_base} is \(\mu_b\) and the mean of
\texttt{BMI\_6yrs} is \(\mu_6\), the difference between
\texttt{BMI\_base} and \texttt{BMI\_6yrs} is \(d_h\). The samples sizes
\(n\) is 10.

The null hypothesis \(H_0\) : \(\mu_b = \mu_6\), the alternative
hypothesis \(H_1\) : \(\mu_b \neq \mu_6\).

\begin{Shaded}
\begin{Highlighting}[]
\NormalTok{d_h =}\StringTok{ }\NormalTok{heavysmoke_df}\OperatorTok{$}\NormalTok{BMI_6yrs }\OperatorTok{-}\StringTok{ }\NormalTok{heavysmoke_df}\OperatorTok{$}\NormalTok{BMI_base }\CommentTok{# get the difference}
\end{Highlighting}
\end{Shaded}

The test statustic : \[t_{s1}= \frac{\bar{d_h}-0}{s_d/\sqrt{n}} \]

\begin{Shaded}
\begin{Highlighting}[]
\NormalTok{t_s1 =}\StringTok{ }\KeywordTok{mean}\NormalTok{(d_h)}\OperatorTok{*}\KeywordTok{sqrt}\NormalTok{(}\DecValTok{10}\NormalTok{)}\OperatorTok{/}\KeywordTok{sd}\NormalTok{(d_h)}
\NormalTok{t_std1 =}\StringTok{ }\KeywordTok{qt}\NormalTok{(}\FloatTok{0.975}\NormalTok{, }\DecValTok{9}\NormalTok{)}
\end{Highlighting}
\end{Shaded}

\(t_{s1}\) is 4.314 and \(t_{9,0.975}\) is 2.262. As \(t_{s1}\) larger
than \(t_{9,0.975}\), we reject \(H_0\).

Interpreatation: At \(\alpha = 0.05\) significant level, we reject
\(H_0\) and conclude that there is enough evidence to prove that the
mean of \texttt{BMI\_base} \(\mu_b\) is different from the mean of
\texttt{BMI\_6yrs} \(\mu_6\) and their BMI has changed 6 years after
quitting smoking.

\subsection{Question 2}\label{question-2-1}

Assuming the difference between \texttt{BMI\_base} and
\texttt{BMI\_6yrs} is never-smoke-group is \(d_n\) and difference
between \(d_h\) and \(d_n\) is \(d\). The samples sizes \(n\) is 10. In
order to compare the BMI changes between women that quit smoking and
women who never smoked, we use two-sample independent t-test to compare
the changes in two groups. First, we need to test the two samples have
same variance or not.

Assuming the variances of two samples are \(\sigma_n\) and \(\sigma_h\)
and the null hypothesis \(H_0\) : \(\sigma_h = \sigma_s\), the
alternative hypothesis \(H_1\) : \(\sigma_h \neq \sigma_n\).

The test statistic: \[ F=\frac{{S_1}^2}{{S_2}^2} \]

\begin{Shaded}
\begin{Highlighting}[]
\NormalTok{d_n =}\StringTok{ }\NormalTok{neversmoke_df}\OperatorTok{$}\NormalTok{BMI_6yrs }\OperatorTok{-}\StringTok{ }\NormalTok{neversmoke_df}\OperatorTok{$}\NormalTok{BMI_base }\CommentTok{# get the difference}
\NormalTok{s_n =}\StringTok{ }\KeywordTok{sd}\NormalTok{(d_n)}
\NormalTok{s_h =}\StringTok{ }\KeywordTok{sd}\NormalTok{(d_h)}
\NormalTok{f =}\StringTok{ }\NormalTok{s_n}\OperatorTok{^}\DecValTok{2}\OperatorTok{/}\NormalTok{s_h}\OperatorTok{^}\DecValTok{2}
\end{Highlighting}
\end{Shaded}

As \(F\) is 0.86 is smaller than \(F_{9,9,0.975}\) 4.026 and is larger
than \(F_{9,9,0.025}\) 0.248, we fail to reject \(H_0\) at
\(\alpha = 0.05\) significant level and conclude that there is no
significant difference between two variances, so we used two-sample
independent t-test with equal variances to compare two differences.

The null hypothesis \(H_0\) : \(d_h = d_s\), the alternative hypothesis
\(H_1\) : \(d_h \neq d_n\).

The test statistic:
\[ t = \frac{\bar{x_1}-\bar{x_2}}{s\sqrt{(\frac{1}{n_1}+\frac{1}{n_2})}} \]
where \(s\) is given by
\[ s^2=\frac{(n_1-1)s_1^2+(n_2-1)s_2^2}{n_1+n_2-2} \]

\begin{Shaded}
\begin{Highlighting}[]
\NormalTok{s =}\StringTok{ }\KeywordTok{sqrt}\NormalTok{(((}\DecValTok{10}\OperatorTok{-}\DecValTok{1}\NormalTok{)}\OperatorTok{*}\NormalTok{s_n}\OperatorTok{^}\DecValTok{2} \OperatorTok{+}\StringTok{ }\NormalTok{(}\DecValTok{10}\OperatorTok{-}\DecValTok{1}\NormalTok{)}\OperatorTok{*}\NormalTok{s_h}\OperatorTok{^}\DecValTok{2}\NormalTok{)}\OperatorTok{/}\NormalTok{(}\DecValTok{10}\OperatorTok{+}\DecValTok{10}\OperatorTok{-}\DecValTok{2}\NormalTok{))}
\NormalTok{t_s2 =}\StringTok{ }\NormalTok{(}\KeywordTok{mean}\NormalTok{(d_n)}\OperatorTok{-}\KeywordTok{mean}\NormalTok{(d_h))}\OperatorTok{/}\NormalTok{(s}\OperatorTok{*}\KeywordTok{sqrt}\NormalTok{(}\DecValTok{1}\OperatorTok{/}\DecValTok{10}\OperatorTok{+}\DecValTok{1}\OperatorTok{/}\DecValTok{10}\NormalTok{))}
\NormalTok{t_std2 =}\StringTok{ }\KeywordTok{qt}\NormalTok{(}\FloatTok{0.975}\NormalTok{, }\DecValTok{18}\NormalTok{)}
\end{Highlighting}
\end{Shaded}

\(t_{s2}\) is -1.704 and \(t_{9,0.975}\) is 2.101. As absolute value of
\(t_{s2}\) smaller than \(t_{9,0.975}\), we fail to reject \(H_0\).

Interpreatation: At \(\alpha = 0.05\) significant level, we fail to
reject \(H_0\) and conclude that there is no enough evidence to prove
that the BMI changes between women that quit smoking and women who never
smoked are different.

\subsection{Question 3}\label{question-3}

The corresponding 95\% CI associated with the difference between changes
of two groups \(d\) is given by:
\[ \bar{x_1}-\bar{x_2}-t_{n_1+n_2-2,1-\alpha/2}s\sqrt{\frac{1}{n_1}+\frac{1}{n_2}} \leq \bar{x_1}-\bar{x_2} \leq \bar{x_1}-\bar{x_2}+t_{n_1+n_2-2,1-\alpha/2}s\sqrt{\frac{1}{n_1}+\frac{1}{n_2}} \]

\begin{Shaded}
\begin{Highlighting}[]
\NormalTok{CI_left =}\StringTok{ }\KeywordTok{mean}\NormalTok{(d_n)}\OperatorTok{-}\KeywordTok{mean}\NormalTok{(d_h) }\OperatorTok{-}\StringTok{ }\KeywordTok{qt}\NormalTok{(}\FloatTok{0.975}\NormalTok{, }\DecValTok{18}\NormalTok{)}\OperatorTok{*}\NormalTok{s}\OperatorTok{*}\KeywordTok{sqrt}\NormalTok{(}\DecValTok{1}\OperatorTok{/}\DecValTok{10}\OperatorTok{+}\DecValTok{1}\OperatorTok{/}\DecValTok{10}\NormalTok{)}
\NormalTok{CI_right =}\StringTok{ }\KeywordTok{mean}\NormalTok{(d_n)}\OperatorTok{-}\KeywordTok{mean}\NormalTok{(d_h) }\OperatorTok{+}\StringTok{ }\KeywordTok{qt}\NormalTok{(}\FloatTok{0.975}\NormalTok{, }\DecValTok{18}\NormalTok{)}\OperatorTok{*}\NormalTok{s}\OperatorTok{*}\KeywordTok{sqrt}\NormalTok{(}\DecValTok{1}\OperatorTok{/}\DecValTok{10}\OperatorTok{+}\DecValTok{1}\OperatorTok{/}\DecValTok{10}\NormalTok{)}
\end{Highlighting}
\end{Shaded}

As \(n\) is 10 and \alpha is 0.05, the corresponding 95\% CI associated
with \(d\) is {[}-4.041, 0.421{]}.

\subsection{Question 4}\label{question-4}

Study design:

We can conduct a cohort study. First, we colloect the BMI of people who
start to quit smoke and the BMI of them 6 years after they quited smoke.
Then we select a group of 100 people from those who able to quit smoke
for at least 6 years including 50 women and 50 men that age 50-64. After
that, we select a group of 100 people including 50 women and 50 men that
had never somke and age 50-64 in the same place as the former group and
record their BMI of first and sixth years. At last, we use two sample
independent t test to test whether their is difference between the
changes of these two groups.

Sample size calculating:

We use the following formula to calculate sample sizes.
\[ n=\frac{(\sigma_1^2+\sigma_2^2)(z_{1-\alpha/2}+z_{1-\beta})^2}{\Delta^2} \]
where \(\Delta=|\mu_2-\mu_1|\).

And we can know that \(\mu_1=3.0\), \(\sigma_1=2.0\), \(\mu_2=1.7\),
\(\sigma_2=1.5\).

\begin{Shaded}
\begin{Highlighting}[]
\NormalTok{mu_}\DecValTok{1}\NormalTok{ =}\StringTok{ }\FloatTok{3.0}
\NormalTok{mu_}\DecValTok{2}\NormalTok{ =}\StringTok{ }\FloatTok{1.7}
\NormalTok{sig_}\DecValTok{1}\NormalTok{ =}\StringTok{ }\FloatTok{2.0}
\NormalTok{sig_}\DecValTok{2}\NormalTok{ =}\StringTok{ }\FloatTok{1.5}
\NormalTok{pow_}\DecValTok{1}\NormalTok{ =}\StringTok{ }\FloatTok{0.8}
\NormalTok{pow_}\DecValTok{2}\NormalTok{ =}\StringTok{ }\FloatTok{0.9}
\NormalTok{signiflevel_}\DecValTok{1}\NormalTok{ =}\StringTok{ }\FloatTok{0.025}
\NormalTok{signiflevel_}\DecValTok{2}\NormalTok{ =}\StringTok{ }\FloatTok{0.05}

\NormalTok{sample_size =}\StringTok{ }\ControlFlowTok{function}\NormalTok{(pow, signiflevel)\{}
\NormalTok{  numerator =}\StringTok{ }\NormalTok{(sig_}\DecValTok{1}\OperatorTok{^}\DecValTok{2}\OperatorTok{+}\NormalTok{sig_}\DecValTok{2}\OperatorTok{^}\DecValTok{2}\NormalTok{)}\OperatorTok{*}\NormalTok{(}\KeywordTok{qnorm}\NormalTok{(}\DecValTok{1}\OperatorTok{-}\NormalTok{signiflevel}\OperatorTok{/}\DecValTok{2}\NormalTok{)}\OperatorTok{+}\KeywordTok{qnorm}\NormalTok{(pow))}\OperatorTok{^}\DecValTok{2}
\NormalTok{  denominator =}\StringTok{ }\NormalTok{(mu_}\DecValTok{1}\OperatorTok{-}\NormalTok{mu_}\DecValTok{2}\NormalTok{)}\OperatorTok{^}\DecValTok{2}
\NormalTok{  n =}\StringTok{ }\NormalTok{numerator}\OperatorTok{/}\NormalTok{denominator}
  \KeywordTok{return}\NormalTok{(n)}
\NormalTok{\}}
\end{Highlighting}
\end{Shaded}

The table of sample sizes are shown below:

\begin{longtable}[]{@{}lll@{}}
\toprule
Choice & 2.5\% significance level & 5\% significance
level\tabularnewline
\midrule
\endhead
80\% power & 35.1517631 & 29.0269221\tabularnewline
90\% power & 45.899434 & 29.0269221\tabularnewline
\bottomrule
\end{longtable}

\section{Problem 3}\label{problem-3}

First, we need to import data ``Knee.csv''.

\begin{Shaded}
\begin{Highlighting}[]
\NormalTok{Knee_df =}\StringTok{ }\KeywordTok{read_csv}\NormalTok{(}\DataTypeTok{file =} \StringTok{"./data/Knee.csv"}\NormalTok{) }\OperatorTok\StringTok{ }
\StringTok{  }\NormalTok{janitor}\OperatorTok{::}\KeywordTok{clean_names}\NormalTok{()}
\end{Highlighting}
\end{Shaded}

\begin{verbatim}
## Parsed with column specification:
## cols(
##   Below = col_integer(),
##   Average = col_integer(),
##   Above = col_integer()
## )
\end{verbatim}

\subsection{Question 1}\label{question-1-2}

The descriptive statistics for Below group is:

\begin{Shaded}
\begin{Highlighting}[]
\KeywordTok{summary}\NormalTok{(Knee_df}\OperatorTok{$}\NormalTok{Below)}
\end{Highlighting}
\end{Shaded}

\begin{verbatim}
## Warning: Unknown or uninitialised column: 'Below'.
\end{verbatim}

\begin{verbatim}
## Length  Class   Mode 
##      0   NULL   NULL
\end{verbatim}

The descriptive statistics for Average group is:

\begin{Shaded}
\begin{Highlighting}[]
\KeywordTok{summary}\NormalTok{(Knee_df}\OperatorTok{$}\NormalTok{Average)}
\end{Highlighting}
\end{Shaded}

\begin{verbatim}
## Warning: Unknown or uninitialised column: 'Average'.
\end{verbatim}

\begin{verbatim}
## Length  Class   Mode 
##      0   NULL   NULL
\end{verbatim}

The descriptive statistics for Above group is:

\begin{Shaded}
\begin{Highlighting}[]
\KeywordTok{summary}\NormalTok{(Knee_df}\OperatorTok{$}\NormalTok{Above)}
\end{Highlighting}
\end{Shaded}

\begin{verbatim}
## Warning: Unknown or uninitialised column: 'Above'.
\end{verbatim}

\begin{verbatim}
## Length  Class   Mode 
##      0   NULL   NULL
\end{verbatim}

\subsection{Question 2}\label{question-2-2}

\begin{Shaded}
\begin{Highlighting}[]
\CommentTok{# tidy data}
\NormalTok{Knee_aov_df =}\StringTok{          }
\StringTok{  }\NormalTok{Knee_df }\OperatorTok\StringTok{ }
\StringTok{  }\KeywordTok{gather}\NormalTok{(}\DataTypeTok{key =} \StringTok{"group"}\NormalTok{, }\DataTypeTok{value =} \StringTok{"time"}\NormalTok{, below}\OperatorTok{:}\NormalTok{above) }\OperatorTok\StringTok{ }
\StringTok{  }\KeywordTok{filter}\NormalTok{(}\OperatorTok{!}\KeywordTok{is.na}\NormalTok{(time))}

\CommentTok{# get ANOVA table}
\NormalTok{res_knee =}\StringTok{ }\KeywordTok{lm}\NormalTok{(time }\OperatorTok{~}\StringTok{ }\KeywordTok{factor}\NormalTok{(group), }\DataTypeTok{data =}\NormalTok{ Knee_aov_df)   }
\KeywordTok{anova}\NormalTok{(res_knee)}
\end{Highlighting}
\end{Shaded}

\begin{verbatim}
## Analysis of Variance Table
## 
## Response: time
##               Df Sum Sq Mean Sq F value    Pr(>F)    
## factor(group)  2 795.25  397.62   19.28 1.454e-05 ***
## Residuals     22 453.71   20.62                      
## ---
## Signif. codes:  0 '***' 0.001 '**' 0.01 '*' 0.05 '.' 0.1 ' ' 1
\end{verbatim}

Assuming the mean of Below group is \(\mu_b\), the mean of Average group
is \(\mu_{av}\) and the mean of Above group is \(\mu_{ab}\).

The null hypothesis \(H_0\) : \(\mu_b = \mu_{av} = \mu_{ab}\), the
alternative hypothesis \(H_1\) : at least two means are not equal.

According to the ANOVA table \(F_s\) is 19.28 and \(F_{2,22,0.99}\) is
5.72. As \(F_s\) larger than \(F_{2,22,0.99}\), we reject \(H_0\).

Interpreatation: At \(\alpha = 0.01\) significant level, we reject
\(H_0\) and conclude that at least two means of Below, Average, Above
groups' time are not equal.

\subsection{Question 3}\label{question-3-1}

\subsubsection{Bonferroni}\label{bonferroni}

\begin{Shaded}
\begin{Highlighting}[]
\KeywordTok{pairwise.t.test}\NormalTok{(Knee_aov_df}\OperatorTok{$}\NormalTok{time, Knee_aov_df}\OperatorTok{$}\NormalTok{group, }\DataTypeTok{p.adj =} \StringTok{'bonferroni'}\NormalTok{)}
\end{Highlighting}
\end{Shaded}

\begin{verbatim}
## 
##  Pairwise comparisons using t tests with pooled SD 
## 
## data:  Knee_aov_df$time and Knee_aov_df$group 
## 
##         above   average
## average 0.0011  -      
## below   1.1e-05 0.0898 
## 
## P value adjustment method: bonferroni
\end{verbatim}

\begin{Shaded}
\begin{Highlighting}[]
\NormalTok{k =}\StringTok{ }\DecValTok{2}
\end{Highlighting}
\end{Shaded}

Bonferroni adjustment:
\[\alpha^* = \frac{\alpha}{\begin{pmatrix} k \\ 2 \\ \end{pmatrix}} \]
In these case, the \(\alpha^*\) is 0.0033333, \(t_{k-1,1-{\alpha^*}}\)
is 12.248. As all \(t\) in t-test table are smaller than
\(t_{k-1,1-{\alpha^*}}\), there is no significance different between
each group.

\subsubsection{Tukey}\label{tukey}

\begin{Shaded}
\begin{Highlighting}[]
\KeywordTok{aov}\NormalTok{(time }\OperatorTok{~}\StringTok{ }\KeywordTok{factor}\NormalTok{(group), }\DataTypeTok{data =}\NormalTok{ Knee_aov_df) }\OperatorTok\StringTok{ }
\StringTok{  }\KeywordTok{TukeyHSD}\NormalTok{()}
\end{Highlighting}
\end{Shaded}

\begin{verbatim}
##   Tukey multiple comparisons of means
##     95% family-wise confidence level
## 
## Fit: aov(formula = time ~ factor(group), data = Knee_aov_df)
## 
## $`factor(group)`
##                    diff        lwr      upr     p adj
## average-above  9.428571  3.8066356 15.05051 0.0010053
## below-above   14.428571  8.5243579 20.33278 0.0000102
## below-average  5.000000 -0.4113011 10.41130 0.0736833
\end{verbatim}

\subsubsection{Dunnett}\label{dunnett}

\begin{Shaded}
\begin{Highlighting}[]
\NormalTok{below_g =}\StringTok{ }\NormalTok{Knee_df}\OperatorTok{$}\NormalTok{below}
\NormalTok{average_g =}\StringTok{ }\NormalTok{Knee_df}\OperatorTok{$}\NormalTok{average}
\NormalTok{above_g =}\StringTok{ }\NormalTok{Knee_df}\OperatorTok{$}\NormalTok{above}

\NormalTok{DescTools}\OperatorTok{::}\KeywordTok{DunnettTest}\NormalTok{(}\KeywordTok{list}\NormalTok{(above_g, average_g, below_g))}
\end{Highlighting}
\end{Shaded}

\begin{verbatim}
## 
##   Dunnett's test for comparing several treatments with a control :  
##     95% family-wise confidence level
## 
## $`1`
##          diff   lwr.ci   upr.ci    pval    
## 2-1  9.428571 4.161393 14.69575 0.00069 ***
## 3-1 14.428571 8.896928 19.96021 6.9e-06 ***
## 
## ---
## Signif. codes:  0 '***' 0.001 '**' 0.01 '*' 0.05 '.' 0.1 ' ' 1
\end{verbatim}

Both of these three pairwise comparisons can test whether there is
significant difference between each two groups in multiple groups.
However, we can find out that the results of each test are different.
The Bonferroni adjustment can test the difference between every two
groups in multiple groups and the result shows that there is no
significant difference in every two groups as the Bonferroni adjustment
is the strictest. The Tukey adjustment can also test the difference
between every two groups in multiple groups and the results shows that
only below and average groups have no significant difference. The
Dunnett adjustment can test the difference between reference
groups(below and average groups) and the control group(above group) and
the result shows that there are significant difference between both
reference groups and control group.

\subsection{Question 4}\label{question-4-1}

At \(\alpha = 0.05\) significant level, we can conclude that when
seperating patients to below, average and above groups, according to
Tukey and Dunnett adjustments both below and average groups are
significant different from above groups in the time required in physical
therapy until successful rehabilitation, which means the time required
in physical therapy is associated with physical status once patients'
physical therapy is above average.

\section{Problem 4}\label{problem-4}

\subsection{Question 1}\label{question-1-3}

\begin{Shaded}
\begin{Highlighting}[]
\NormalTok{UCBA_df =}\StringTok{ }\KeywordTok{as.tibble}\NormalTok{(datasets}\OperatorTok{::}\NormalTok{UCBAdmissions) }\OperatorTok\StringTok{     }\CommentTok{# import data}
\StringTok{  }\NormalTok{janitor}\OperatorTok{::}\KeywordTok{clean_names}\NormalTok{()}

\NormalTok{admit_male =}\StringTok{                          }\CommentTok{# get the number of admitted male in each department}
\StringTok{  }\NormalTok{UCBA_df }\OperatorTok\StringTok{ }
\StringTok{  }\KeywordTok{filter}\NormalTok{(admit }\OperatorTok{==}\StringTok{ "Admitted"}\NormalTok{, gender }\OperatorTok{==}\StringTok{ "Male"}\NormalTok{)}

\NormalTok{admit_female =}\StringTok{                        }\CommentTok{# get the number of admitted female in each department}
\StringTok{  }\NormalTok{UCBA_df }\OperatorTok\StringTok{ }
\StringTok{  }\KeywordTok{filter}\NormalTok{(admit }\OperatorTok{==}\StringTok{ "Admitted"}\NormalTok{, gender }\OperatorTok{==}\StringTok{ "Female"}\NormalTok{)}

\NormalTok{reject_male =}\StringTok{                          }\CommentTok{# get the number of rejected male in each department}
\StringTok{  }\NormalTok{UCBA_df }\OperatorTok\StringTok{ }
\StringTok{  }\KeywordTok{filter}\NormalTok{(admit }\OperatorTok{==}\StringTok{ "Rejected"}\NormalTok{, gender }\OperatorTok{==}\StringTok{ "Male"}\NormalTok{)}

\NormalTok{reject_female =}\StringTok{                        }\CommentTok{# get the number of rejected female in each department}
\StringTok{  }\NormalTok{UCBA_df }\OperatorTok\StringTok{ }
\StringTok{  }\KeywordTok{filter}\NormalTok{(admit }\OperatorTok{==}\StringTok{ "Rejected"}\NormalTok{, gender }\OperatorTok{==}\StringTok{ "Female"}\NormalTok{)}

\NormalTok{x_m =}\StringTok{ }\KeywordTok{sum}\NormalTok{(admit_male}\OperatorTok{$}\NormalTok{n)}
\NormalTok{x_f =}\StringTok{ }\KeywordTok{sum}\NormalTok{(admit_female}\OperatorTok{$}\NormalTok{n)}

\NormalTok{n_f =}\StringTok{ }\KeywordTok{sum}\NormalTok{(admit_female}\OperatorTok{$}\NormalTok{n) }\OperatorTok{+}\StringTok{ }\KeywordTok{sum}\NormalTok{(reject_female}\OperatorTok{$}\NormalTok{n)}
\NormalTok{n_m =}\StringTok{ }\KeywordTok{sum}\NormalTok{(admit_male}\OperatorTok{$}\NormalTok{n) }\OperatorTok{+}\StringTok{ }\KeywordTok{sum}\NormalTok{(reject_male}\OperatorTok{$}\NormalTok{n)}

\NormalTok{p_m =}\StringTok{ }\NormalTok{x_m}\OperatorTok{/}\NormalTok{n_m}
\NormalTok{p_f =}\StringTok{ }\NormalTok{x_f}\OperatorTok{/}\NormalTok{n_f}
\end{Highlighting}
\end{Shaded}

Using the point estimation of the proportions of female and male
admitted at Berkeley. The point estimation of the proportions of female
\(\hat{p_f}\) is 0.304, the point estimation of the proportions of male
\(\hat{p_m}\) is 0.445.

Using the following formula to get the 95\% confidence interval for the
proportions of female and male:
\[ \left(\hat{p}-z_{0.975}\sqrt{\frac{\hat{p}(1-\hat{p})}{n}},\hat{p}+z_{0.975}\sqrt{\frac{\hat{p}(1-\hat{p})}{n}}\right) \]

\begin{Shaded}
\begin{Highlighting}[]
\CommentTok{# proportion of female}
\NormalTok{left_CI_female =}\StringTok{ }\NormalTok{p_f }\OperatorTok{-}\StringTok{ }\KeywordTok{qnorm}\NormalTok{(}\FloatTok{0.975}\NormalTok{)}\OperatorTok{*}\KeywordTok{sqrt}\NormalTok{(p_f}\OperatorTok{*}\NormalTok{(}\DecValTok{1}\OperatorTok{-}\NormalTok{p_f)}\OperatorTok{/}\NormalTok{n_f)}
\NormalTok{right_CI_female =}\StringTok{ }\NormalTok{p_f }\OperatorTok{+}\StringTok{ }\KeywordTok{qnorm}\NormalTok{(}\FloatTok{0.975}\NormalTok{)}\OperatorTok{*}\KeywordTok{sqrt}\NormalTok{(p_f}\OperatorTok{*}\NormalTok{(}\DecValTok{1}\OperatorTok{-}\NormalTok{p_f)}\OperatorTok{/}\NormalTok{n_f)}

\CommentTok{# proportion of male}
\NormalTok{left_CI_male =}\StringTok{ }\NormalTok{p_m }\OperatorTok{-}\StringTok{ }\KeywordTok{qnorm}\NormalTok{(}\FloatTok{0.975}\NormalTok{)}\OperatorTok{*}\KeywordTok{sqrt}\NormalTok{(p_m}\OperatorTok{*}\NormalTok{(}\DecValTok{1}\OperatorTok{-}\NormalTok{p_m)}\OperatorTok{/}\NormalTok{n_m)}
\NormalTok{right_CI_male =}\StringTok{ }\NormalTok{p_m }\OperatorTok{+}\StringTok{ }\KeywordTok{qnorm}\NormalTok{(}\FloatTok{0.975}\NormalTok{)}\OperatorTok{*}\KeywordTok{sqrt}\NormalTok{(p_m}\OperatorTok{*}\NormalTok{(}\DecValTok{1}\OperatorTok{-}\NormalTok{p_m)}\OperatorTok{/}\NormalTok{n_m)}
\end{Highlighting}
\end{Shaded}

By using the above formula, we can get the 95\% confidence interval for
the proportions of female is ( 0.283, 0.325 ) and the 95\% confidence
interval for the proportions of male is ( 0.426, 0.464 )

According to the mean of two proportions, we can find that The point
estimation of the proportions of female \(\hat{p_f}\) is lightly smaller
than the point estimation of the proportions of male \(\hat{p_m}\) as
well as the confidence interval which might indicates that the true
proportions of female admitted in Berkeley is smaller than the true
proportions of male admitted in Berkeley.

\subsection{Question 2}\label{question-2-3}

The null hypothesis \(H_0\) : \(p_f = p_m\), the alternative hypothesis
\(H_1\) : \(p_f \neq p_m\). The test statustuc with continuity
correction is given by:
\[ z = \frac{|\hat{p_1}-\hat{p_2}|-(\frac{1}{2n_1}+\frac{1}{2n_2})}{\sqrt{\hat{p}\hat{q}(\frac{1}{n_1}+\frac{1}{n_2})}} \]
when \(\hat{p} = \frac{n_1\hat{p_1}+n_2\hat{p_2}}{n_1+n_2}\).

We create a function \texttt{z.prop} to calculate test statistic. The
function is shown below.

\begin{Shaded}
\begin{Highlighting}[]
\NormalTok{z.prop =}\StringTok{ }\ControlFlowTok{function}\NormalTok{(x1,x2,n1,n2)\{}
\NormalTok{  numerator =}\StringTok{ }\KeywordTok{abs}\NormalTok{((x1}\OperatorTok{/}\NormalTok{n1) }\OperatorTok{-}\StringTok{ }\NormalTok{(x2}\OperatorTok{/}\NormalTok{n2)) }\OperatorTok{-}\StringTok{ }\NormalTok{(}\DecValTok{1}\OperatorTok{/}\NormalTok{(}\DecValTok{2}\OperatorTok{*}\NormalTok{n1)}\OperatorTok{+}\DecValTok{1}\OperatorTok{/}\NormalTok{(}\DecValTok{2}\OperatorTok{*}\NormalTok{n2))}
\NormalTok{  p.common =}\StringTok{ }\NormalTok{(x1}\OperatorTok{+}\NormalTok{x2) }\OperatorTok{/}\StringTok{ }\NormalTok{(n1}\OperatorTok{+}\NormalTok{n2)}
\NormalTok{  denominator =}\StringTok{ }\KeywordTok{sqrt}\NormalTok{(p.common }\OperatorTok{*}\StringTok{ }\NormalTok{(}\DecValTok{1}\OperatorTok{-}\NormalTok{p.common) }\OperatorTok{*}\StringTok{ }\NormalTok{(}\DecValTok{1}\OperatorTok{/}\NormalTok{n1 }\OperatorTok{+}\StringTok{ }\DecValTok{1}\OperatorTok{/}\NormalTok{n2))}
\NormalTok{  z.prop.ris =}\StringTok{ }\NormalTok{numerator }\OperatorTok{/}\StringTok{ }\NormalTok{denominator}
  \KeywordTok{return}\NormalTok{(z.prop.ris)}
\NormalTok{\}}

\NormalTok{z_stat =}\StringTok{ }\KeywordTok{z.prop}\NormalTok{(x_f, x_m, n_f, n_m)}
\end{Highlighting}
\end{Shaded}

By calculating, we can know,\(z\) is 9.571 and \(z_{1-\alpha/2}\) is
1.96. \(z\) is larger than \(z_{1-\alpha/2}\) and p-value is 1.

Interpretation: At \(\alpha = 0.05\) significant level, we reject
\(H_0\) and conclude that their are significant difference between the
true proportions of female admitted in Berkeley and the true proportions
of male admitted in Berkeley.


\end{document}
